\documentclass[]{article}
\usepackage{lmodern}
\usepackage{amssymb,amsmath}
\usepackage{ifxetex,ifluatex}
\usepackage{fixltx2e} % provides \textsubscript
\ifnum 0\ifxetex 1\fi\ifluatex 1\fi=0 % if pdftex
  \usepackage[T1]{fontenc}
  \usepackage[utf8]{inputenc}
\else % if luatex or xelatex
  \ifxetex
    \usepackage{mathspec}
  \else
    \usepackage{fontspec}
  \fi
  \defaultfontfeatures{Ligatures=TeX,Scale=MatchLowercase}
\fi
% use upquote if available, for straight quotes in verbatim environments
\IfFileExists{upquote.sty}{\usepackage{upquote}}{}
% use microtype if available
\IfFileExists{microtype.sty}{%
\usepackage{microtype}
\UseMicrotypeSet[protrusion]{basicmath} % disable protrusion for tt fonts
}{}
\usepackage[margin=1in]{geometry}
\usepackage{hyperref}
\hypersetup{unicode=true,
            pdftitle={Susliks in Miroslav July 2019},
            pdfauthor={Fernando Mateos-Gonzalez},
            pdfborder={0 0 0},
            breaklinks=true}
\urlstyle{same}  % don't use monospace font for urls
\usepackage{color}
\usepackage{fancyvrb}
\newcommand{\VerbBar}{|}
\newcommand{\VERB}{\Verb[commandchars=\\\{\}]}
\DefineVerbatimEnvironment{Highlighting}{Verbatim}{commandchars=\\\{\}}
% Add ',fontsize=\small' for more characters per line
\usepackage{framed}
\definecolor{shadecolor}{RGB}{248,248,248}
\newenvironment{Shaded}{\begin{snugshade}}{\end{snugshade}}
\newcommand{\AlertTok}[1]{\textcolor[rgb]{0.94,0.16,0.16}{#1}}
\newcommand{\AnnotationTok}[1]{\textcolor[rgb]{0.56,0.35,0.01}{\textbf{\textit{#1}}}}
\newcommand{\AttributeTok}[1]{\textcolor[rgb]{0.77,0.63,0.00}{#1}}
\newcommand{\BaseNTok}[1]{\textcolor[rgb]{0.00,0.00,0.81}{#1}}
\newcommand{\BuiltInTok}[1]{#1}
\newcommand{\CharTok}[1]{\textcolor[rgb]{0.31,0.60,0.02}{#1}}
\newcommand{\CommentTok}[1]{\textcolor[rgb]{0.56,0.35,0.01}{\textit{#1}}}
\newcommand{\CommentVarTok}[1]{\textcolor[rgb]{0.56,0.35,0.01}{\textbf{\textit{#1}}}}
\newcommand{\ConstantTok}[1]{\textcolor[rgb]{0.00,0.00,0.00}{#1}}
\newcommand{\ControlFlowTok}[1]{\textcolor[rgb]{0.13,0.29,0.53}{\textbf{#1}}}
\newcommand{\DataTypeTok}[1]{\textcolor[rgb]{0.13,0.29,0.53}{#1}}
\newcommand{\DecValTok}[1]{\textcolor[rgb]{0.00,0.00,0.81}{#1}}
\newcommand{\DocumentationTok}[1]{\textcolor[rgb]{0.56,0.35,0.01}{\textbf{\textit{#1}}}}
\newcommand{\ErrorTok}[1]{\textcolor[rgb]{0.64,0.00,0.00}{\textbf{#1}}}
\newcommand{\ExtensionTok}[1]{#1}
\newcommand{\FloatTok}[1]{\textcolor[rgb]{0.00,0.00,0.81}{#1}}
\newcommand{\FunctionTok}[1]{\textcolor[rgb]{0.00,0.00,0.00}{#1}}
\newcommand{\ImportTok}[1]{#1}
\newcommand{\InformationTok}[1]{\textcolor[rgb]{0.56,0.35,0.01}{\textbf{\textit{#1}}}}
\newcommand{\KeywordTok}[1]{\textcolor[rgb]{0.13,0.29,0.53}{\textbf{#1}}}
\newcommand{\NormalTok}[1]{#1}
\newcommand{\OperatorTok}[1]{\textcolor[rgb]{0.81,0.36,0.00}{\textbf{#1}}}
\newcommand{\OtherTok}[1]{\textcolor[rgb]{0.56,0.35,0.01}{#1}}
\newcommand{\PreprocessorTok}[1]{\textcolor[rgb]{0.56,0.35,0.01}{\textit{#1}}}
\newcommand{\RegionMarkerTok}[1]{#1}
\newcommand{\SpecialCharTok}[1]{\textcolor[rgb]{0.00,0.00,0.00}{#1}}
\newcommand{\SpecialStringTok}[1]{\textcolor[rgb]{0.31,0.60,0.02}{#1}}
\newcommand{\StringTok}[1]{\textcolor[rgb]{0.31,0.60,0.02}{#1}}
\newcommand{\VariableTok}[1]{\textcolor[rgb]{0.00,0.00,0.00}{#1}}
\newcommand{\VerbatimStringTok}[1]{\textcolor[rgb]{0.31,0.60,0.02}{#1}}
\newcommand{\WarningTok}[1]{\textcolor[rgb]{0.56,0.35,0.01}{\textbf{\textit{#1}}}}
\usepackage{graphicx,grffile}
\makeatletter
\def\maxwidth{\ifdim\Gin@nat@width>\linewidth\linewidth\else\Gin@nat@width\fi}
\def\maxheight{\ifdim\Gin@nat@height>\textheight\textheight\else\Gin@nat@height\fi}
\makeatother
% Scale images if necessary, so that they will not overflow the page
% margins by default, and it is still possible to overwrite the defaults
% using explicit options in \includegraphics[width, height, ...]{}
\setkeys{Gin}{width=\maxwidth,height=\maxheight,keepaspectratio}
\IfFileExists{parskip.sty}{%
\usepackage{parskip}
}{% else
\setlength{\parindent}{0pt}
\setlength{\parskip}{6pt plus 2pt minus 1pt}
}
\setlength{\emergencystretch}{3em}  % prevent overfull lines
\providecommand{\tightlist}{%
  \setlength{\itemsep}{0pt}\setlength{\parskip}{0pt}}
\setcounter{secnumdepth}{0}
% Redefines (sub)paragraphs to behave more like sections
\ifx\paragraph\undefined\else
\let\oldparagraph\paragraph
\renewcommand{\paragraph}[1]{\oldparagraph{#1}\mbox{}}
\fi
\ifx\subparagraph\undefined\else
\let\oldsubparagraph\subparagraph
\renewcommand{\subparagraph}[1]{\oldsubparagraph{#1}\mbox{}}
\fi

%%% Use protect on footnotes to avoid problems with footnotes in titles
\let\rmarkdownfootnote\footnote%
\def\footnote{\protect\rmarkdownfootnote}

%%% Change title format to be more compact
\usepackage{titling}

% Create subtitle command for use in maketitle
\providecommand{\subtitle}[1]{
  \posttitle{
    \begin{center}\large#1\end{center}
    }
}

\setlength{\droptitle}{-2em}

  \title{Susliks in Miroslav July 2019}
    \pretitle{\vspace{\droptitle}\centering\huge}
  \posttitle{\par}
    \author{Fernando Mateos-Gonzalez}
    \preauthor{\centering\large\emph}
  \postauthor{\par}
    \date{}
    \predate{}\postdate{}
  

\begin{document}
\maketitle

This report analises data from capture-recapture sessions during 15-19
July 2019. The aim is to estimate population characteristics (mainly
density) in Miroslav airport, to be able to extrapolate and apply to
other areas of the Czech Republic.

We will follow the protocols described here
\url{https://www.otago.ac.nz/density/}

Useful links: Example analysis
\url{https://www.otago.ac.nz/density/pdfs/secr-tutorial.pdf}

The following boxes include all the code and results to run the analyses
in R. To install and run models in secr, you must download the package
and load it. In the next box, we install and load the package secr, to
do our analyses, and the package ``here'', to find files.

\begin{Shaded}
\begin{Highlighting}[]
\KeywordTok{library}\NormalTok{(secr)}
\end{Highlighting}
\end{Shaded}

\begin{verbatim}
## This is secr 3.2.1. For overview type ?secr
\end{verbatim}

\begin{Shaded}
\begin{Highlighting}[]
\KeywordTok{library}\NormalTok{(here)}
\end{Highlighting}
\end{Shaded}

\begin{verbatim}
## here() starts at D:/FPI/2019/CZ/SOUSLIK/susliks_july_2019
\end{verbatim}

Now we create the capthist, the file combining our captures with the
trap locations:

\begin{Shaded}
\begin{Highlighting}[]
\NormalTok{july <-}\StringTok{ }\KeywordTok{read.capthist}\NormalTok{(}\KeywordTok{here}\NormalTok{(}\StringTok{"data"}\NormalTok{, }\StringTok{"captures.txt"}\NormalTok{), }\KeywordTok{here}\NormalTok{(}\StringTok{"data"}\NormalTok{, }\StringTok{"traps.txt"}\NormalTok{), }\DataTypeTok{detector =} \StringTok{"proximity"}\NormalTok{)}
\end{Highlighting}
\end{Shaded}

\begin{verbatim}
## Session miroslav 
## More than one detection per detector per occasion at binary detector(s)
\end{verbatim}

\begin{Shaded}
\begin{Highlighting}[]
\KeywordTok{summary}\NormalTok{(july)}
\end{Highlighting}
\end{Shaded}

\begin{verbatim}
## Object class       capthist 
## Detector type      proximity 
## Detector number    20 
## Average spacing    13.57938 m 
## x-range            -623778.5 -623653.2 m 
## y-range            -1187164 -1187091 m 
## 
## Counts by occasion 
##                    1  2  3  4  5 Total
## n                 23 27 16 19  8    93
## u                 23 22 12 11  5    73
## f                 58 10  5  0  0    73
## M(t+1)            23 45 57 68 73    73
## losses             0  0  0  0  0     0
## detections        24 27 17 21  8    97
## detectors visited 14 12 12 15  8    61
## detectors used    20 20 20 20 20   100
\end{verbatim}

n number of distinct individuals detected on each occasion t u number of
individuals detected for the first time on each occasion t f number of
individuals detected on exactly t occasions M(t+1) cumulative number of
detected individuals on each occasion t

Now we use the plot method, which for capthist objects has additional
arguments; we set tracks = TRUE to join consecutive captures of each
individual. (use arguments gridl and gridsp to suppress the grid or vary
its spacing, and border to ``zoom'' in the area where the traps are).

\begin{Shaded}
\begin{Highlighting}[]
\KeywordTok{par}\NormalTok{(}\DataTypeTok{mar =} \KeywordTok{c}\NormalTok{(}\DecValTok{1}\NormalTok{,}\DecValTok{1}\NormalTok{,}\DecValTok{1}\NormalTok{,}\DecValTok{1}\NormalTok{)) }\CommentTok{# reduce margins}
\KeywordTok{plot}\NormalTok{ (july, }\DataTypeTok{tracks =} \OtherTok{TRUE}\NormalTok{, }\DataTypeTok{gridsp =} \DecValTok{5}\NormalTok{,}\DataTypeTok{border =} \DecValTok{10}\NormalTok{)}
\end{Highlighting}
\end{Shaded}

\begin{verbatim}
## Warning in plot.capthist(july, tracks = TRUE, gridsp = 5, border = 10):
## track for repeat detections on same occasion joins points in arbitrary
## sequence
\end{verbatim}

\includegraphics{susliks_july_2019_files/figure-latex/unnamed-chunk-3-1.pdf}

The most important insight from this figure is that individuals tend to
be recaptured near their site of first capture. This is expected when
the individuals of a species occupy home ranges. In SECR models the
tendency for detections to be localised is reflected in the spatial
scale parameter σ. Good estimation of σ and density D requires spatial
recaptures (i.e.~captures at sites other than the site of first
capture).

Successive trap-revealed movements can be extracted with the moves
function and summarised with hist:

\begin{Shaded}
\begin{Highlighting}[]
\NormalTok{m <-}\StringTok{ }\KeywordTok{unlist}\NormalTok{(}\KeywordTok{moves}\NormalTok{(july))}
\KeywordTok{par}\NormalTok{(}\DataTypeTok{mar =} \KeywordTok{c}\NormalTok{(}\FloatTok{3.2}\NormalTok{,}\DecValTok{4}\NormalTok{,}\DecValTok{1}\NormalTok{,}\DecValTok{1}\NormalTok{), }\DataTypeTok{mgp =} \KeywordTok{c}\NormalTok{(}\FloatTok{2.1}\NormalTok{,}\FloatTok{0.6}\NormalTok{,}\DecValTok{0}\NormalTok{)) }\CommentTok{# reduce margins}
\KeywordTok{hist}\NormalTok{(m, }\DataTypeTok{breaks =} \KeywordTok{seq}\NormalTok{(}\DecValTok{0}\OperatorTok{/}\DecValTok{2}\NormalTok{, }\DecValTok{30}\NormalTok{,}\DecValTok{2}\NormalTok{), }\DataTypeTok{xlab =} \StringTok{"Movement m"}\NormalTok{, }\DataTypeTok{main =} \StringTok{""}\NormalTok{)}
\end{Highlighting}
\end{Shaded}

\includegraphics{susliks_july_2019_files/figure-latex/unnamed-chunk-4-1.pdf}

The function RPSV with option CC = TRUE provides a biased estimate of
the spatial scale σ, ignoring the problem that movements are truncated
by the edge of the grid:

\begin{Shaded}
\begin{Highlighting}[]
\NormalTok{initialsigma <-}\StringTok{ }\KeywordTok{RPSV}\NormalTok{(july, }\DataTypeTok{CC =} \OtherTok{TRUE}\NormalTok{)}
\KeywordTok{cat}\NormalTok{(}\StringTok{"Quick and biased estimate of sigma ="}\NormalTok{, initialsigma, }\StringTok{"m}\CharTok{\textbackslash{}n}\StringTok{"}\NormalTok{)}
\end{Highlighting}
\end{Shaded}

\begin{verbatim}
## Quick and biased estimate of sigma = 5.116154 m
\end{verbatim}

This estimate will be useful when we come to fit a model.

Next we fit the simplest possible SECR model with function secr.fit.
Setting trace = FALSE suppresses printing of intermediate likelihood
evaluations; it doesn't hurt to leave it out. We save the fitted model
with the name `fit'.

To examine model output or extract particular results you should use one
of the functions defined for the purpose. Technically, these are S3
methods for the class `secr'. The key methods are print, plot, AIC,
coef, vcov and predict. Append `.secr' when seeking help
e.g.~?print.secr. Typing the name of the fitted model at the R prompt
invokes the print method for secr objects and displays a more useful
report.

\begin{Shaded}
\begin{Highlighting}[]
\NormalTok{fit <-}\StringTok{ }\KeywordTok{secr.fit}\NormalTok{ (july, }\DataTypeTok{buffer =} \DecValTok{4} \OperatorTok{*}\StringTok{ }\NormalTok{initialsigma, }\DataTypeTok{trace =} \OtherTok{FALSE}\NormalTok{,}\DataTypeTok{biasLimit =} \OtherTok{NA}\NormalTok{, }\DataTypeTok{verify =} \OtherTok{FALSE}\NormalTok{)}

\KeywordTok{detector}\NormalTok{(}\KeywordTok{traps}\NormalTok{(july)) <-}\StringTok{ "proximity"}

\NormalTok{fit}
\end{Highlighting}
\end{Shaded}

\begin{verbatim}
## 
## secr.fit(capthist = july, buffer = 4 * initialsigma, verify = FALSE, 
##     biasLimit = NA, trace = FALSE)
## secr 3.2.1, 12:40:05 22 Jul 2019
## 
## Detector type      proximity 
## Detector number    20 
## Average spacing    13.57938 m 
## x-range            -623778.5 -623653.2 m 
## y-range            -1187164 -1187091 m 
## 
## N animals       :  73  
## N detections    :  97 
## N occasions     :  5 
## Mask area       :  0.9380393 ha 
## 
## Model           :  D~1 g0~1 sigma~1 
## Fixed (real)    :  none 
## Detection fn    :  halfnormal
## Distribution    :  poisson 
## N parameters    :  3 
## Log likelihood  :  -198.4705 
## AIC             :  402.9409 
## AICc            :  403.2888 
## 
## Beta parameters (coefficients) 
##            beta   SE.beta       lcl       ucl
## D      5.560902 0.1998769  5.169151  5.952654
## g0    -1.705310 0.2997312 -2.292772 -1.117848
## sigma  1.825563 0.1107730  1.608452  2.042674
## 
## Variance-covariance matrix of beta parameters 
##                  D          g0        sigma
## D      0.039950765 -0.02430287 -0.007066575
## g0    -0.024302868  0.08983879 -0.019885810
## sigma -0.007066575 -0.01988581  0.012270666
## 
## Fitted (real) parameters evaluated at base levels of covariates 
##        link   estimate SE.estimate          lcl         ucl
## D       log 260.057389 52.50296021 175.76552388 384.7731007
## g0    logit   0.153773  0.03900308   0.09172332   0.2464107
## sigma   log   6.206287  0.68960371   4.99507138   7.7112017
\end{verbatim}

The report comprises these sections that you should identify: • function
call and time stamp • summary of the data • description of the model,
including the maximized log likelihood, Akaike's Information Criterion
AIC • estimates of model coefficients (beta parameters) • estimates of
variance-covariance matrix of the coefficients • estimates of the `real'
parameters

The last three items are generated by the coef, vcov and predict methods
respectively. The final table of estimates is the most interesting, but
it is derived from the other two. For our simple model there is one beta
parameter for each real parameter (We can get from beta parameter
estimates to real parameter estimates by applying the inverse of the
link function e.g.: Dˆ = exp(βˆD), and similarly for confidence limits;
standard errors require a delta-method approximation (Lebreton et
al.~1992).)

. The estimated density is 260 susliks per hectare, 95\% confidence
interval 175-384 susliks per hectare . The other two real parameters
jointly determine the detection function that you can easily plot with
95\% confidence limits

\begin{Shaded}
\begin{Highlighting}[]
\KeywordTok{par}\NormalTok{(}\DataTypeTok{mar =} \KeywordTok{c}\NormalTok{(}\DecValTok{4}\NormalTok{,}\DecValTok{4}\NormalTok{,}\DecValTok{1}\NormalTok{,}\DecValTok{1}\NormalTok{)) }\CommentTok{# reduce margins}
\KeywordTok{plot}\NormalTok{(fit, }\DataTypeTok{limits =} \OtherTok{TRUE}\NormalTok{)}
\end{Highlighting}
\end{Shaded}

\includegraphics{susliks_july_2019_files/figure-latex/unnamed-chunk-7-1.pdf}

The theory of SECR tells us that buffer width is not critical as long as
it is wide enough that animals at the edge have effectively zero chance
of appearing in our sample. The 4 suggestion is based on experience with
half-normal detection models7. We check that for the present model with
the function esa.plot. The estimated density8 has easily reached a
plateau at the chosen buffer width (dashed red line):

\begin{Shaded}
\begin{Highlighting}[]
\KeywordTok{esa.plot}\NormalTok{(fit)}
\KeywordTok{abline}\NormalTok{(}\DataTypeTok{v =} \DecValTok{4} \OperatorTok{*}\StringTok{ }\NormalTok{initialsigma, }\DataTypeTok{lty =} \DecValTok{2}\NormalTok{, }\DataTypeTok{col =} \StringTok{'red'}\NormalTok{)}
\end{Highlighting}
\end{Shaded}

\includegraphics{susliks_july_2019_files/figure-latex/unnamed-chunk-8-1.pdf}

\hypertarget{choosing-a-detection-function}{%
\subsection{Choosing a detection
function}\label{choosing-a-detection-function}}

• detection probability declines with distance according to a
half-normal curve

We can try alternative shapes for the detection function (the decline in
detection probability with distance). secr offers several different
shapes of detection function (see the list at ?detectfn). We need to
sort these out. All except ANN and HAN decline monotonically with
distance. Three are only used for acoustic data (BSS, SS, SSS). The
simplest UN is not available for maximum likelihood model fitting, and
several are frankly exotic and almost never used (CHN, WEX, CLN, CG), as
are ANN and HAN.

That leaves the half-normal, negative exponential, and hazard rate
functions (HN, EX, HR) These differ primarily in the length of their
tails i.e.~the probability they assign to very distant detections. The
half-normal makes distant detections very improbable, the negative
exponential less so; The `hazard-rate' function requires a third
parameter and potentially has a very long tail indeed. Fit each of these
and assess the effect. We use a wider buffer to allow for longer tails

\begin{Shaded}
\begin{Highlighting}[]
\NormalTok{fit.HN <-}\StringTok{ }\KeywordTok{secr.fit}\NormalTok{ (july, }\DataTypeTok{buffer =} \DecValTok{6} \OperatorTok{*}\StringTok{ }\NormalTok{initialsigma, }\DataTypeTok{detectfn =} \StringTok{'HN'}\NormalTok{, }\DataTypeTok{trace =} \OtherTok{FALSE}\NormalTok{, }\DataTypeTok{biasLimit =} \OtherTok{NA}\NormalTok{,}\DataTypeTok{verify =} \OtherTok{FALSE}\NormalTok{)}
\NormalTok{fit.EX <-}\StringTok{ }\KeywordTok{secr.fit}\NormalTok{ (july, }\DataTypeTok{buffer =} \DecValTok{6} \OperatorTok{*}\StringTok{ }\NormalTok{initialsigma, }\DataTypeTok{detectfn =} \StringTok{'EX'}\NormalTok{, }\DataTypeTok{trace =} \OtherTok{FALSE}\NormalTok{, }\DataTypeTok{biasLimit =} \OtherTok{NA}\NormalTok{,}\DataTypeTok{verify =} \OtherTok{FALSE}\NormalTok{)}
\NormalTok{fit.HR <-}\StringTok{ }\KeywordTok{secr.fit}\NormalTok{ (july, }\DataTypeTok{buffer =} \DecValTok{6} \OperatorTok{*}\StringTok{ }\NormalTok{initialsigma, }\DataTypeTok{detectfn =} \StringTok{'HR'}\NormalTok{, }\DataTypeTok{trace =} \OtherTok{FALSE}\NormalTok{, }\DataTypeTok{biasLimit =} \OtherTok{NA}\NormalTok{,}\DataTypeTok{verify =} \OtherTok{FALSE}\NormalTok{)}

\NormalTok{fits <-}\StringTok{ }\KeywordTok{secrlist}\NormalTok{(}\DataTypeTok{HN =}\NormalTok{ fit.HN, }\DataTypeTok{EX =}\NormalTok{ fit.EX, }\DataTypeTok{HR =}\NormalTok{ fit.HR)}
\KeywordTok{predict}\NormalTok{(fits)}
\end{Highlighting}
\end{Shaded}

\begin{verbatim}
## $HN
##        link    estimate SE.estimate          lcl         ucl
## D       log 259.0853043 52.63609966 174.68563216 384.2628273
## g0    logit   0.1534672  0.03899836   0.09145387   0.2461387
## sigma   log   6.2218485  0.70141024   4.99187120   7.7548873
## 
## $EX
##        link    estimate SE.estimate         lcl         ucl
## D       log 283.4497841 65.34727060 181.4566459 442.7712180
## g0    logit   0.3537688  0.09537974   0.1946272   0.5535912
## sigma   log   3.9093957  0.55930652   2.9576508   5.1674034
## 
## $HR
##        link    estimate SE.estimate          lcl         ucl
## D       log 274.5515082 66.89732832 171.47583915 439.5868889
## g0    logit   0.1805899  0.08336792   0.06807976   0.3993572
## sigma   log   5.4087399  2.46704600   2.30695116  12.6810085
## z       log   3.3653260  1.09322446   1.80892318   6.2608624
\end{verbatim}

When everything is done, we upload it to github following
\url{https://happygitwithr.com/existing-github-first.html}


\end{document}
